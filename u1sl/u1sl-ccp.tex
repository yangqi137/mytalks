\documentclass[xcolor=table, 10pt, aspectratio=169]{beamer}

%\usepackage{arev}
\usepackage{amsmath,amssymb,amscd}
\usepackage{dsfont}
\usepackage{mathrsfs}
\usepackage{yfonts}
\usepackage{pifont}
\usepackage{bm}
\usepackage{graphicx}
\usepackage{tabularx}
\usepackage{animate}
%\usepackage{ifthen}

%\usepackage{xeCJK}
%\usepackage{fontspec}
%\newfontfamily\cjkfont{PingFang SC}
%\setCJKmainfont{PingFang SC}
\newcolumntype{x}{>{\centering\arraybackslash}X}
\renewcommand{\arraystretch}{1.5}

\usepackage{tikz}
	\usetikzlibrary{calc}
	\usetikzlibrary{arrows,shapes, positioning, matrix}
	\usetikzlibrary{decorations.markings}
	\tikzstyle arrowstyle=[scale=1]
	\tikzstyle directed=[postaction={decorate,decoration={markings,
 	   mark=at position .15 with {\arrow[arrowstyle]{stealth}}}}]
\tikzstyle string=[thick,postaction={decorate,decoration={markings,
    mark=at position .55 with {\arrow[arrowstyle]{stealth}}}}]
\tikzstyle dual_string=[dashed,postaction={decorate,decoration={markings,
    mark=at position .55 with {\arrow[arrowstyle]{stealth}}}}]

\tikzstyle dw=[thick,postaction={decorate,decoration={markings,
    mark=at position 1 with {\arrow[arrowstyle]{stealth}}}}]
\tikzstyle group=[mbg]

\usepackage{pgffor}
\newcommand{\mb}[1]{\mathbf{#1}}
\renewcommand{\cal}[1]{\mathcal{#1}}

\newcommand{\ag}[2]{#1_\mb{#2}}
\newcommand{\cohosub}[1]{\scalebox{0.72}{\textswab{#1}}}
\newcommand{\cohosubsub}[1]{\scalebox{0.6}{\textswab{#1}}}
\newcommand{\coho}[1]{\textswab{#1}}


\mode<presentation>
{
  %\usetheme{Warsaw}
  % or ...
  %\useoutertheme{rectangle}
  \setbeamertemplate{frametitle}[default][center]
  \defbeamertemplate{itemize item}{flat}{\begin{pgfpicture}{-1ex}{0ex}{1ex}{2ex}
      \pgfpathcircle{\pgfpoint{0pt}{.6ex}}{0.6ex}
      \pgfusepath{fill}
    \end{pgfpicture}%
  }
  \defbeamertemplate{itemize subitem}{flat}{\footnotesize\raise0.5pt\hbox{\textbullet}}
  \defbeamertemplate{itemize subsubitem}{flat}{\footnotesize\raise0.5pt\hbox{\textbullet}}

  %\useinnertheme{circles}
  \setbeamertemplate{items}[flat]
  \setbeamertemplate{sections/subsections in toc}[circle]
  \setbeamertemplate{blocks}[rounded]
  \setbeamertemplate{title page}[default][colsep=-4bp,rounded=true]
  \setbeamertemplate{part page}[default][colsep=-4bp,rounded=true]
  \setbeamercovered{transparent}
  %\usecolortheme{spruce}
  %\definecolor{THU}{RGB}{116,61,130}
  \definecolor{mbg}{RGB}{0,0,160}
  \setbeamercolor*{palette primary}{fg=white,bg=mbg}
  \setbeamercolor*{titlelike}{parent=palette primary}
  \setbeamercolor*{structure}{fg=mbg}
  \setbeamercolor{frametitle}{fg=white,bg=mbg}
  % or whatever (possibly just delete it)
  \setbeamercolor{block title}{bg=mbg,fg=white}
  \setbeamercolor{block body}{bg=mbg!15}


  \addtobeamertemplate{navigation symbols}{}{ \hspace{1em}%
    \usebeamerfont{footline}%
    \insertframenumber / \inserttotalframenumber }
}


%\usepackage[english]{babel}
% or whatever

%\usepackage[latin1]{inputenc}
% or whatever

%\usepackage{times}
%\usepackage[T1]{fontenc}
% Or whatever. Note that the encoding and the font should match. If T1
% does not look nice, try deleting the line with the fontenc.

\title[U1SL] % (optional, use only with long paper titles)
{Monte Carlo Study of Compact Quantum QED with Fermionic Matter [2+1D U(1)-Dirac Spin Liquid]}

\author[Y Qi] % (optional, use only with lots of authors)
{Yang~Qi}
% - Give the names in the same order as the appear in the paper.
% - Use the \inst{?} command only if the authors have different
%   affiliation.

\institute[Fudan] % (optional, but mostly needed)
{
Department of Physics, Fudan University.
}
% - Use the \inst command only if there are several affiliations.
% - Keep it simple, no one is interested in your street address.

%\date{2016 Annual Meeting of Fudan CFTPP} % (optional, should be abbreviation of conference name)
%{Fudan University, Oct 13 2015}
\date{CCP2019, CUHK, July 2019.}
% - Either use conference name or its abbreviation.
% - Not really informative to the audience, more for people (including
%   yourself) who are reading the slides online

\subject{Theoretical Physics}
% This is only inserted into the PDF information catalog. Can be left
% out.



% If you have a file called "university-logo-filename.xxx", where xxx
% is a graphic format that can be processed by latex or pdflatex,
% resp., then you can add a logo as follows:

\pgfdeclareimage[height=1cm]{university-logo}{../resources/fudan}
\logo{\pgfuseimage{university-logo}}



% Delete this, if you do not want the table of contents to pop up at
% the beginning of each subsection:
\AtBeginSection[]
{
  \begin{frame}<beamer>{Outline}
			\tableofcontents[currentsection,currentsubsection]
  \end{frame}
}
%\AtBeginSubsection[]
%{
 % \begin{frame}<beamer>{Outline}
  %  \tableofcontents[currentsection,currentsubsection]
  %\end{frame}
%}


\begin{document}

\begin{frame}
  \titlepage
\end{frame}

\begin{frame}{Collaborators}
\begin{itemize}
\item Xiao Yan Xu, Hong Kong University of Science and Technology.
\item Zi Yang Meng, Hong Kong University; Institute of Physics, Beijing.
\item Long Zhang, Kavli Institute for Theoretical Sciences, UCAS, Beijing.
\item Fakher F. Assaad, Universit\"at W\"urzburg.
\item Cenke Xu, University of California, Santa Barbara.
\end{itemize}
\begin{center}
  \includegraphics[height=2cm]{../people/xiaoyanxu}
  \includegraphics[height=2cm]{../people/ziyangmeng}
  \includegraphics[height=2cm]{../people/zhanglong}
  \includegraphics[height=2cm]{../people/fakher}
  \includegraphics[height=2cm]{../people/cenke}
\end{center}
\begin{center}
  \small Phys. Rev. X \textbf{9}, 021022 (2019).
\end{center}
\end{frame}

\begin{frame}{Outline}
	%\begin{columns}
	%\column{.7\textwidth}
		\tableofcontents
  %\end{columns}
  % You might wish to add the option [pausesections]
\end{frame}

\section{Introduction}

\begin{frame}
  \frametitle{U(1) Quantum Spin Liquid: History and Outlooks}
\begin{itemize}
	\item U(1)-Dirac spin liquid: Dirac-fermion spinons + U(1) gauge field.
	\[\mathcal L = \bar\psi_\alpha \gamma^\mu(\partial_\mu-iA_\mu)\psi +  F_{\mu\nu}^2,\quad
	\vec S_i = \psi^\dagger_{i\alpha}\vec\sigma_{\alpha\beta}\psi_{i\beta}.\]
	\item $\pi$-flux phase in high-$T_c$ cuprates:\\
	\emph{\small I. Affleck and J. B. Marston, Phys. Rev. B 37, 3774 (1988).}
	\item Spin liquid in frustrated magnets: kagome-lattice Heisenberg model?\\
	\emph{\small Y. Ran, M. Hermele, P. A. Lee, and X.-G. Wen, Phys. Rev. Lett. 98, 117205 (2007).}

  \item Does the mean-field picture stand: Is the U(1) gauge field deconfined?
\end{itemize}
\end{frame}

%\begin{frame}
%  \frametitle{Compact U(1): Monopoles}
%  \begin{itemize}
%    \item Gauge theory in the continuum:
%    \[S=\int d^3x (\epsilon_{abc}A_c)^2.\]
%    \item Gauge theory on a lattice:
%    \[S=\sum_{\square}e^{iA_{ij}}e^{iA_{jk}}e^{iA_{kl}}e^{iA_{li}}.\]
%    \item Compact U(1) gauge connection: $A_{ij}\simeq A_{ij}+2\pi$.
%    \item Monopole: non-smooth space-time configurations of $A_{ij}$.
%    \item Even without matter field, compact U(1) is not free Maxwell theory: confinement in the IR.
%  \end{itemize}
%\end{frame}
\begin{frame}
  \frametitle{Confinement}
  \begin{columns}
    \column{.6\textwidth}
    \begin{itemize}
      \item Gauge force between two test charges: $V(r)$.
      \begin{itemize}
        \item $V(r)\leq C$ as $r\rightarrow \infty$: like $V(r)\simeq -\frac1r$: deconfined.
        \item $V(r)\rightarrow\infty$ as $r\rightarrow \infty$: like $V(r)\simeq r$: confined.
      \end{itemize}
      \item Example: quark confinement.
      \item Confinement is bad:
      $\vec S=f^\dagger \vec\sigma f$ does not work / fluctuation beyond mean field is too strong.
    \end{itemize}
    \column{.4\textwidth}
    \begin{center}
      \includegraphics[width=.8\textwidth]{QCD-confinement}
    \end{center}
  \end{columns}
\end{frame}

\begin{frame}
	\frametitle{Does U(1) spin liquid exist in 2+1D?}
	\[\mathcal L = \bar\psi_\alpha \gamma^\mu(\partial_\mu-iA_\mu)\psi + F_{\mu\nu}^2.\]
	\begin{itemize}
		\item U(1) monopole: confinement. Compact U(1) is \alert{not free}.
		\item Gapless fermions suppress monopole.
	\end{itemize}
	\begin{tabular}{p{.3\textwidth}p{.3\textwidth}p{.3\textwidth}}
		\includegraphics[scale=.45]{balance1}
		&
		\includegraphics[scale=.45]{balance2}
		&
		\includegraphics[scale=.45]{balance3}\\
		Pure compact U(1). & Spinon FS. & A critical $N_f$?\\
		A. M. Polyakov, Gauge fields and strings. & Sung-Sik Lee, PRB 78 085129 (2008). & Under heavy debate: large-$N_f$ expansion / CFT argument / numerical simulations.
	\end{tabular}
\end{frame}

\begin{frame}
\frametitle{Designer Hamiltonians}
\begin{columns}
	\column{.45\textwidth}
	\begin{center}
		\includegraphics[width=\columnwidth]{../orthogonal_metal/berg2012}
	\end{center}
	\column{.55\textwidth}
	\begin{itemize}
		\item E Berg, M Metlitski and S Sachdev,\\ Science 2012.
		\item Use a two-band model to simulate SDW instability in cuprates.
		\item Two-band model removes sign problem.
	\end{itemize}
\end{columns}
\end{frame}

\begin{frame}
	\frametitle{Simulating effective models of spin liquids}
		\begin{center}
			\includegraphics[width=4cm]{../orthogonal_metal/z2dsl}
		\end{center}
		\begin{itemize}
			\item F Assaad and T Grover, PRX 2016
			\item S Gazit, M Randeria and A Vishwanath, Nat Phys 2017
			\item Simulated $\mathbb Z_2$-Dirac spin liquid
		\end{itemize}
\end{frame}

\begin{frame}
	\frametitle{Our Approach}
	\[H = \sum_{ij}\vec S_i\cdot\vec S_j+\cdots
	\quad\Rightarrow\quad
	\mathcal L = \bar\psi_\alpha \gamma^\mu(\partial_\mu-iA_\mu)\psi
	+ F_{\mu\nu}^2.
	\quad\Rightarrow\quad\text{Confinement??}\]
	\begin{itemize}
		\item[\ding{51}] Realize the QED$_3$ theory in the IR.
		\item[\ding{51}] Universal properties of QED$_3$ in the IR
		\item[\ding{55}] Which realistic model realizes it?
		\item[\ding{55}] Which material realizes it?
	\end{itemize}
\end{frame}

\section{Model and Method}

\begin{frame}
  \frametitle{The Model Hamiltonian}
  \[
  H=\frac{1}{2}JN_{f}\sum_{\langle i,j \rangle} \frac 1 4 \hat{L}^{2}_{ij}-t\sum_{\langle i,j \rangle\alpha}\left(\hat{c}^{\dagger}_{i\alpha}e^{i\hat{\theta}_{ij}}\hat{c}_{j\alpha}+\text{h.c.}\right)
  +\ \frac{1}{2}K\ N_f\sum_{\square}\cos \left( \text{curl} \hat{\theta} \right).
\]
\begin{columns}
  \column{.8\textwidth}
  \begin{itemize}
    \item $c$: spinon.
    \item $\theta_{ij}$: quantum rotors $\Rightarrow$ compact gauge field.
    \item $[L_{ij}, e^{i\theta_{ij}}]=e^{i\theta_{ij}}$ is the angular momentum / electric field.
    \item Gauge symmetry:
    \[\hat{Q}_{i} = -\sum_{j}\hat{L}_{ij} + \sum_{\alpha} \left( \hat{c}^{\dagger}_{i\alpha}\hat{c}^{\phantom\dagger}_{i\alpha} - 1/2 \right);[H, \hat Q_i] = 0.\]
    \item Dynamically generated constraint:
		$\hat Q_i \simeq 0$ implies gauge symmetry.
    \item $K>0$ favors a $\pi$-flux state.
		\item $N_f=1$: two Dirac cones (two valley). $N_f=2,4,6,\ldots$ do not have sign problem.
  \end{itemize}

  \column{.2\textwidth}
  \begin{center}
    \includegraphics[width=3cm]{model}
  \end{center}
\end{columns}
\end{frame}

%\begin{frame}
%  \frametitle{Dynamically generated constraint}
%  \[C_{Q} = \frac{1}{L^2} \sum_{ij} \langle \hat{Q}_i \hat{Q}_j\rangle,\quad C_Q\rightarrow0\text{ as }\beta\propto L\rightarrow0\].
%  \begin{center}
%    \includegraphics[width=.5\textwidth]{n2QQ00}
%
%		$Q\rightarrow 0$: emergent gauge symmetry.
%  \end{center}
%\end{frame}

\begin{frame}
  \frametitle{The Determinant Quantum Monte Carlo Method}
  \begin{align*}
  L_F &= \sum_{\langle i,j \rangle\alpha}{\psi}^{\dagger}_{i\alpha} \left[(\partial_\tau -\mu)\delta_{ij}-t e^{i\phi_{ij}}   \right]   {\psi}_{j\alpha} + \text{h.c.}\\
  L_\phi &= \frac{4} {JN_{f}\Delta \tau ^2} \sum_{\langle i,j \rangle}
  \left[ 1-\cos(\phi_{ij}(\tau+1)-\phi_{ij}(\tau))
   +\frac{1}{2}K N_f\sum_{\square}\cos (\text{curl} \phi) \right],
\end{align*}
\begin{itemize}
  \item Sampling bosonic configuration: $\phi = \{\phi_{ij}(\tau)\}$.
  \item Weight:
  \[W[\phi] = W_b[\phi]\det(\mathbf I + \mathbf B(\beta,0;\phi))\]
  \item Complexity: $O(\beta N^3)$.
  \item Metropolis Algorithm and Fast Update:
  \[\phi_{ij}(\tau)\rightarrow\phi_{ij}(\tau)+\delta\phi;\quad p(\delta\phi)\propto e^{-\delta\phi^2/(2\Delta^2)}\].
\end{itemize}
\end{frame}

\section{Results}

\begin{frame}
  \frametitle{Phase diagrams}
  \begin{center}
    \includegraphics[width=5cm]{phase-diagram}
  \end{center}
  \begin{columns}[t]
    \column{.5\textwidth}
    \begin{block}{U(1)-Dirac Spin Liquid}
      \begin{itemize}
        \item Deconfined U(1) gauge field.
        \item Power-law spin-spin and dimer-dimer correlation functions.
        \item Algebraic spin liquid.
      \end{itemize}
    \end{block}

    \column{.5\textwidth}
    \begin{block}{Confinement phases}
      \begin{itemize}
        \item VBS or AFM long-range order.
        \item Fermion gapped.
        \item Confined U(1) gauge field.
      \end{itemize}
    \end{block}
  \end{columns}
\end{frame}

\begin{frame}
  \frametitle{$N_f=2$: phase boundary}
  \begin{columns}
		\column{.7\textwidth}
    \includegraphics[width=\textwidth]{n2rafm}
		\column{.3\textwidth}
		\begin{itemize}
			\item Correlation ratio: $r = 1-\frac{\chi(\bm Q+\delta\bm q)}{\chi(\bm Q)}.$
			\item Crossing indicates the position of phase transition.
			\item Phase transition b/w U1D and AFM: $J_c=1.6(2)$.
		\end{itemize}
  \end{columns}
\end{frame}

\begin{frame}
  \frametitle{Deconfined U(1) gauge field}
  \begin{columns}
		\column{.6\textwidth}
    \includegraphics[width=\textwidth]{photonmass}
		\column{.4\textwidth}
		\begin{itemize}
			\item Measure photon mass $m$ from decaying of gauge-flux correlation functions.
			\item $m > 0$: confinement phase.
		\end{itemize}
  \end{columns}
\end{frame}

\begin{frame}
  \frametitle{$N_f=2$: U1D phase}
  \begin{columns}
		\column{.5\textwidth}
    \includegraphics[width=\textwidth]{n2decay}
		\column{.5\textwidth}
		\begin{itemize}
			\item Spin and dimer OPs are fermion bilinears:
			\[\vec S = \psi^\dagger_{i\alpha}\vec\sigma_{\alpha\beta}\psi_{i\beta},\]
			\[D_{x,y} = \psi^\dagger_{i\alpha}\psi_{i+x,y,\beta}.\]
			\item Algebraic spin liquid: $\Delta_S = \Delta_D$.
		  \item (Free fermion has $\Delta_S=\Delta_D = 4$.)
		\end{itemize}
  \end{columns}
\end{frame}

\begin{frame}
  \frametitle{$N_f=4$: phase boundary}
  \begin{columns}
		\column{.7\textwidth}
    \includegraphics[width=\textwidth]{n4rvbs}
		\column{.3\textwidth}
		\begin{itemize}
			\item Phase transition b/w U1D and VBS: $J_c=1.2(3)$.
		\end{itemize}
  \end{columns}
\end{frame}

\begin{frame}
  \frametitle{$N_f=4$: U1D phase}
  \begin{columns}
		\column{.5\textwidth}
    \includegraphics[width=\textwidth]{n4decay}
		\column{.5\textwidth}
		\begin{itemize}
			\item Spin and dimer OPs are fermion bilinears:
			\[\vec S = \psi^\dagger_{i\alpha}\vec\sigma_{\alpha\beta}\psi_{i\beta},\]
			\[D_{x,y} = \psi^\dagger_{i\alpha}\psi_{i+x,y,\beta}.\]
			\item Algebraic spin liquid: $\Delta_S = \Delta_D$.
		  \item (Free fermion has $\Delta_S=\Delta_D = 4$.)
		\end{itemize}
  \end{columns}
\end{frame}

\begin{frame}
  \frametitle{$N_f=6$: phase boundary}
  \begin{columns}
		\column{.7\textwidth}
    \includegraphics[width=\textwidth]{n6rvbs}
		\column{.3\textwidth}
		\begin{itemize}
			\item Phase transition b/w U1D and VBS: $J_c=1.9(3)$.
		\end{itemize}
  \end{columns}
\end{frame}

\begin{frame}
  \frametitle{$N_f=6$: U1D phase}
  \begin{columns}
		\column{.5\textwidth}
    \includegraphics[width=\textwidth]{n6decay}
		\column{.5\textwidth}
		\begin{itemize}
			\item Spin and dimer OPs are fermion bilinears:
			\[\vec S = \psi^\dagger_{i\alpha}\vec\sigma_{\alpha\beta}\psi_{i\beta},\]
			\[D_{x,y} = \psi^\dagger_{i\alpha}\psi_{i+x,y,\beta}.\]
			\item Algebraic spin liquid: $\Delta_S = \Delta_D$.
		  \item (Free fermion has $\Delta_S=\Delta_D = 4$.)
		\end{itemize}
  \end{columns}
\end{frame}

\begin{frame}
  \frametitle{$N_f=8$: phase boundary}
  \begin{columns}
		\column{.7\textwidth}
    \includegraphics[width=\textwidth]{n8rvbs}
		\column{.3\textwidth}
		\begin{itemize}
			\item Phase transition b/w U1D and VBS: $J_c=2.5(1)$.
		\end{itemize}
  \end{columns}
\end{frame}

\begin{frame}
  \frametitle{$N_f=8$: U1D phase}
  \begin{columns}
		\column{.5\textwidth}
    \includegraphics[width=\textwidth]{n8decay}
		\column{.5\textwidth}
		\begin{itemize}
			\item Spin and dimer OPs are fermion bilinears:
			\[\vec S = \psi^\dagger_{i\alpha}\vec\sigma_{\alpha\beta}\psi_{i\beta},\]
			\[D_{x,y} = \psi^\dagger_{i\alpha}\psi_{i+x,y,\beta}.\]
			\item Algebraic spin liquid: $\Delta_S = \Delta_D$.
		  \item (Free fermion has $\Delta_S=\Delta_D = 4$.)
		\end{itemize}
  \end{columns}
\end{frame}

\begin{frame}
  \frametitle{Summary}
  \begin{columns}
		\column{.7\textwidth}
    \includegraphics[width=\textwidth]{eta}
		\column{.3\textwidth}
		$\Delta_S=\Delta_D$ consistent with large-$N_f$-expansion results.
  \end{columns}
\end{frame}

\section{Conclusion}

\begin{frame}{Conclusion and outlooks}
  \begin{center}
    \includegraphics[width=5cm]{phase-diagram}
  \end{center}
  \begin{itemize}
    \item Deconfined U1D ASL phase observed for $N_f=2,4,6,8$.
    \item Confinement/deconfinement transition: continuous transition? universality class?
    \item Phase transition between VBS and AFM for $N_f=4$.
    \item Will U1D phase be stable in the IR limit? Larger sizes / directly measure monopole-operator scaling dimensions.
    \item Improving the algorithm: combining HMC / DQMC?
  \end{itemize}
\end{frame}

\end{document}
