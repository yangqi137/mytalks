\documentclass[xcolor=table, 11pt, aspectratio=169]{beamer}

%\usepackage{arev}
\usepackage{amsmath,amssymb,amscd}
\usepackage{dsfont}
\usepackage{mathrsfs}
\usepackage{yfonts}
\usepackage{bm}
\usepackage{graphicx}
\usepackage{tabularx}
\usepackage{animate}
\usepackage{listings}
\usepackage{pifont}
%\usepackage{mathtools}
%\usepackage{ifthen}

%\usepackage{xeCJK}
%\usepackage{fontspec}
%\newfontfamily\cjkfont{PingFang SC}
%\setCJKmainfont{PingFang SC}
\newcolumntype{x}{>{\centering\arraybackslash}X}
\renewcommand{\arraystretch}{1.5}
\newcommand{\uone}{\mathrm U(1)}
\DeclareMathOperator{\img}{img}
\lstset{language=GAP}

\usepackage{tikz}
	\usetikzlibrary{calc}
	\usetikzlibrary{arrows,shapes, positioning, matrix}
	\usetikzlibrary{decorations.markings}
	\tikzset{>=stealth}
	\tikzstyle arrowstyle=[scale=1]
	\tikzstyle directed=[postaction={decorate,decoration={markings,
 	   mark=at position .15 with {\arrow[arrowstyle]{stealth}}}}]
\tikzstyle string=[thick,postaction={decorate,decoration={markings,
    mark=at position .55 with {\arrow[arrowstyle]{stealth}}}}]
\tikzstyle dual_string=[dashed,postaction={decorate,decoration={markings,
    mark=at position .55 with {\arrow[arrowstyle]{stealth}}}}]

\tikzstyle dw=[thick,postaction={decorate,decoration={markings,
    mark=at position 1 with {\arrow[arrowstyle]{stealth}}}}]
\tikzstyle group=[mbg]
\newcommand*{\halfway}{0.5*\pgfdecoratedpathlength+.5*8pt}\tikzstyle arrowstyle=[scale=1]
\newcommand*{\halfwayb}{0.5*\pgfdecoratedpathlength}
\tikzstyle arrowstyle=[scale=1]
\tikzstyle fermion=[thick,postaction={decorate},decoration={markings,
    mark=at position \halfway with {\arrow[arrowstyle]{latex}}}]
\tikzstyle fermion2=[thick,postaction={decorate},decoration={markings,
        mark=at position \halfwayb with {\arrow[arrowstyle]{latex}}}]

\usepackage{pgffor}
\newcommand{\mb}[1]{\mathbf{#1}}
\renewcommand{\cal}[1]{\mathcal{#1}}

\newcommand{\ag}[2]{#1_\mb{#2}}
\newcommand{\cohosub}[1]{\scalebox{0.72}{\textswab{#1}}}
\newcommand{\cohosubsub}[1]{\scalebox{0.6}{\textswab{#1}}}
\newcommand{\coho}[1]{\textswab{#1}}

\DeclareMathOperator{\tr}{Tr}
\DeclareMathOperator{\im}{Im}
\DeclareMathOperator{\re}{Re}

\mode<presentation>
{
  %\usetheme{Warsaw}
  % or ...
  %\useoutertheme{rectangle}
  \setbeamertemplate{frametitle}[default][center]
  \defbeamertemplate{itemize item}{flat}{\begin{pgfpicture}{-1ex}{0ex}{1ex}{2ex}
      \pgfpathcircle{\pgfpoint{0pt}{.6ex}}{0.6ex}
      \pgfusepath{fill}
    \end{pgfpicture}%
  }
  \defbeamertemplate{itemize subitem}{flat}{\footnotesize\raise0.5pt\hbox{\textbullet}}
  \defbeamertemplate{itemize subsubitem}{flat}{\footnotesize\raise0.5pt\hbox{\textbullet}}

  %\useinnertheme{circles}
  \setbeamertemplate{items}[flat]
  \setbeamertemplate{sections/subsections in toc}[circle]
  \setbeamertemplate{blocks}[rounded]
  \setbeamertemplate{title page}[default][colsep=-4bp,rounded=true]
  \setbeamertemplate{part page}[default][colsep=-4bp,rounded=true]
  \setbeamercovered{transparent}
  %\usecolortheme{spruce}
  %\definecolor{THU}{RGB}{116,61,130}
  \definecolor{mbg}{RGB}{0,0,160}
  \setbeamercolor*{palette primary}{fg=white,bg=mbg}
  \setbeamercolor*{titlelike}{parent=palette primary}
  \setbeamercolor*{structure}{fg=mbg}
  \setbeamercolor{frametitle}{fg=white,bg=mbg}
  % or whatever (possibly just delete it)
  \setbeamercolor{block title}{bg=mbg,fg=white}
  \setbeamercolor{block body}{bg=mbg!15}


  \addtobeamertemplate{navigation symbols}{}{ \hspace{1em}%
    \usebeamerfont{footline}%
    \insertframenumber / \inserttotalframenumber }
}


%\usepackage[english]{babel}
% or whatever

%\usepackage[latin1]{inputenc}
% or whatever

%\usepackage{times}
%\usepackage[T1]{fontenc}
% Or whatever. Note that the encoding and the font should match. If T1
% does not look nice, try deleting the line with the fontenc.

\title[Space-group SPTs] % (optional, use only with long paper titles)
{Classification of Topological Crystalline States}

\author[Y Qi] % (optional, use only with lots of authors)
{Yang~Qi}
% - Give the names in the same order as the appear in the paper.
% - Use the \inst{?} command only if the authors have different
%   affiliation.

\institute[Fudan] % (optional, but mostly needed)
{Department of Physics, Fudan University}
% - Use the \inst command only if there are several affiliations.
% - Keep it simple, no one is interested in your street address.

%\date{2016 Annual Meeting of Fudan CFTPP} % (optional, should be abbreviation of conference name)
\date{WENG, Xi'an, April 2024.}
% - Either use conference name or its abbreviation.
% - Not really informative to the audience, more for people (including
%   yourself) who are reading the slides online

\subject{Theoretical Physics}
% This is only inserted into the PDF information catalog. Can be left
% out.



% If you have a file called "university-logo-filename.xxx", where xxx
% is a graphic format that can be processed by latex or pdflatex,
% resp., then you can add a logo as follows:

\pgfdeclareimage[height=1cm]{university-logo}{../resources/fudan}
\logo{\pgfuseimage{university-logo}}

\AtBeginSection[]
{
\begin{frame}{Outline}
%	\begin{columns}
%		\begin{column}[t]{.45\textwidth}
%		\begin{center}
%			\includegraphics[height=4cm]{toys}
%		\end{center}
%	\end{column}
%	\begin{minipage}[t][0.5\textheight]{0.55\textwidth}
      \tableofcontents[currentsection]
%    \end{minipage}\hfill
%	\end{columns}
\end{frame}
}


% Delete this, if you do not want the table of contents to pop up at
% the beginning of each subsection:

\begin{document}

\begin{frame}
  \titlepage
\end{frame}

\begin{frame}{Collaborators}
  \begin{itemize}
  \item Tian Yuan: Fudan University.
  \item Chen Fang: Institute of Physics, CAS.
\end{itemize}
\vspace{4em}
\begin{center}
        \includegraphics[height=3.5cm]{../people/tianyuan}~~~~
      \includegraphics[height=3.5cm]{../people/chenfang}
    \end{center}
\end{frame}

\begin{frame}{Outline}
%	\begin{columns}
%		\begin{column}[t]{.45\textwidth}
%		\begin{center}
%			\includegraphics[height=4cm]{toys}
%		\end{center}
%	\end{column}
%	\begin{minipage}[t][0.5\textheight]{0.55\textwidth}
      \tableofcontents
%    \end{minipage}\hfill
%	\end{columns}
\end{frame}


\section{Introduction to Topological Crystalline States (TCSs)}

\begin{frame}
  \frametitle{Symmetry-Protected Topological (SPT) phases}
  \begin{itemize}
  \item Landau paragidm: phases are classified by symmetry breaking.
    \begin{center}
      \includegraphics[height=1.5cm]{../resources/crystal}~~
      \includegraphics[height=1.5cm]{../resources/magnet}~~
      \includegraphics[height=1.5cm]{../resources/sc}
    \end{center}
  \item SPT: gapped topological phases beyond Landau paradiam.
  \item Gapped bulk : cannot be smoothly connected to a trivial state without closing gap or breaking symmetry.
  \item Symmetry-protected gapless surface states.
  \item Example: integer quantum Hall states; topological insulators.
    \begin{center}
      \includegraphics[height=2cm]{../resources/fqhe}~~
      \includegraphics[height=2cm]{ti_surface}
    \end{center}
  \end{itemize}
\end{frame}

\begin{frame}
  \frametitle{Abelian-group classification}
  \begin{itemize}
  \item SPT phases and boundary anomalies are classified by Abelian groups ($\mathbb Z$ or $\mathbb Z_n$).
    \begin{itemize}
    \item Addition: stacking of phases/gapless boundaries.
    \item 0: The trivial phase/gapped boundary.
    \end{itemize}
  \item Classification: determined by symmetry group $G$ and dimension $d$.
  \item 2D Chern-insulators (Integer Quantum Hall):
    \begin{center}
      \includegraphics[width=8cm]{qhe_edge}
    \end{center}
    Classified by $\mathbb Z$: $[n]+[m]=[n+m]$; $[n]+[-n] = 0$.
  \end{itemize}
\end{frame}

\begin{frame}
	\frametitle{Abelian-group classification}
	\begin{itemize}
		\item 3D Topological Insulators:
		\begin{center}
			\includegraphics[width=8cm]{ti_surface}
		\end{center}
		Classified by $\mathbb Z_2$: $[1]+[1] = 0$.
		\item 1D Haldane chain:
		\begin{center}
			\includegraphics[width=6cm]{../dimer/weak3d_aklt_blue}
		\end{center}
		Classified by $\mathbb Z_2$: $[1]+[1] = 0$.
	\end{itemize}
\end{frame}

\begin{frame}
    \frametitle{Topological Crystaline States = Space-group SPT}
    \begin{itemize}
    \item Two approaches:
      \begin{enumerate}
      \item Thorngren and Else (2018): the crystalline equivalence principle
        \[SG\simeq G;\quad \Phi^d(SG)\simeq\Phi^d(G).\]
      \item Real-space recipes: Zhida Song, Chen Fang and YQ, Nat. Commun. (2020).\\
        \emph{Examples: mirror SPT, weak SPT (translation symmetry).}
        %\emph{Patch construction: Zhida Song, Shengjie Huang, YQ, Chen Fang and Michael Hermele, Sci. Adv. 5, eaax2007 (2019).}
      \end{enumerate}
    \item Application to free-fermion systems?
    \end{itemize}
    \begin{center}
    \begin{tikzpicture}[scale=.9]
    \fill [blue!20] (0,0)--(1,1)--(1,3)--(0,2)--(0,0);
    \draw (0,0)--(0,2)--(1,3);
    \draw (-1.5,0)--(1.5,0)--(1.5,2)--(-1.5,2)--(-1.5,0);
    \draw (1.5,0)--(2.5,1)--(2.5,3)--(1.5,2);
    \draw (2.5,3)--(-.5,3)--(-1.5,2);
    \end{tikzpicture}
    \hspace{2em}
    \begin{tikzpicture}[scale=.9]
    \fill [blue!40,opacity=.5] (0,0)--(1,1)--(1,3)--(0,2)--(0,0);
    \draw (0,0)--(0,2)--(1,3);
    \fill [blue!40,opacity=.5] (.5,0)--(1.5,1)--(1.5,3)--(0.5,2)--(0.5,0);
    \draw (.5,0)--(.5,2)--(1.5,3);
    \fill [blue!40,opacity=.5] (1,0)--(2,1)--(2,3)--(1,2)--(1,0);
    \draw (1,0)--(1,2)--(2,3);
    \fill [blue!40,opacity=.5] (-.5,0)--(.5,1)--(.5,3)--(-0.5,2)--(-0.5,0);
    \draw (-.5,0)--(-.5,2)--(.5,3);
    \fill [blue!40,opacity=.5] (-1,0)--(0,1)--(0,3)--(-1,2)--(-1,0);
    \draw (-1,0)--(-1,2)--(0,3);
    \draw (-1.5,0)--(1.5,0)--(1.5,2)--(-1.5,2)--(-1.5,0);
    \draw (1.5,0)--(2.5,1)--(2.5,3)--(1.5,2);
    \draw (2.5,3)--(-.5,3)--(-1.5,2);
    \end{tikzpicture}
    \end{center}
    \end{frame}
    
    \begin{frame}{Why Topological Crystaline States?}
      \begin{itemize}
        \item Crystalline symmetries are present in all solid-state materials.
        \item High-order boundary states: TCSs (excluding the Atomic Insulators) are a.k.a. High-Order Topological States.
        May be useful for storing quantum information.
      \end{itemize}
      \begin{center}
        \begin{tikzpicture} [scale=2]
          \fill [blue!50] (0,0)--(1,0)--(1.5,.5)--(1.5,1.5)--(0.5,1.5)--(0,1)--(0,0);
          \draw [thick,red] (0,0)--(1,0)--(1,1)--(0,1)--(0,0);
          \draw [thick,red] (1,0)--(1.5,.5)--(1.5,1.5)--(1,1);
          \draw [thick,red] (1.5,1.5)--(0.5,1.5)--(0,1);
          \draw [thick,red] (0,0)--(.5,.5)--(.5,1.5);
          \draw [thick,red](.5,.5)--(1.5,.5);
          \fill [red] (0,0) circle (2pt);
          \fill [red] (1,0) circle (2pt);
          \fill [red] (0,1) circle (2pt);
          \fill [red] (1,1) circle (2pt);
          \fill [red] (0.5,0.5) circle (2pt);
          \fill [red] (1.5,0.5) circle (2pt);
          \fill [red] (0.5,1.5) circle (2pt);
          \fill [red] (1.5,1.5) circle (2pt);
        \end{tikzpicture}      
      \end{center}
    \end{frame}

    \begin{frame}{K Theory 101: 10-fold way}
          \begin{columns}
    \column{.4\textwidth}
    \begin{itemize}
      \item 10-fold way and Bott periodicity.
      \item TRS: $T^2=\pm1$;
      \item PHS: $C^2=\pm1$;
      \item Chiral symmetry: $S=TC$.
    \end{itemize}
    \column{.6\textwidth}{\small
    \begin{tabular}{c|ccc|cc|cccc}
      \hline
      Class & $T^2$ & $C^2$ & $S$ & Type & q & 0d & 1d & 2d & 3d\\
      \hline\hline
      A & 0 & 0 & 0 & $\mathbb C$ & 0 & $\mathbb Z$ & 0 & $\mathbb Z$ & 0\\
      AIII & 0&0&1 & $\mathbb C$&1 & 0 & $\mathbb Z$ & 0 & $\mathbb Z$\\
      \hline
      D & 0&+1&0 & $\mathbb R$&0 & $\mathbb Z_2$&$\mathbb Z_2$&$\mathbb Z$&0\\
      DIII & $-1$&+1&1 & $\mathbb R$&1 & 0&$\mathbb Z_2$&$\mathbb Z_2$&$\mathbb Z$\\
      AII & $-1$&0&0 & $\mathbb R$&2 & $\mathbb Z$&0&$\mathbb Z_2$&$\mathbb Z_2$\\
      CII & $-1$&$-1$&1 & $\mathbb R$&3 & 0&$\mathbb Z$&0&$\mathbb Z_2$\\
      C & 0&$-1$&0 & $\mathbb R$&4 & 0&0&$\mathbb Z$&0\\
      CI & +1&$-1$&1 & $\mathbb R$&5 & 0&0&0&$\mathbb Z$\\
      AI & +1&0&0 & $\mathbb R$&6 & $\mathbb Z$&0&0&0\\
      BDI & +1&+1&1 & $\mathbb R$&7 & $\mathbb Z_2$&$\mathbb Z$&0&0\\
      \hline
    \end{tabular}}
  \end{columns}
    \end{frame}

  \begin{frame}{K Theory 101: Atiyah-Bott-Shapiro (ABS) construction}
    \begin{itemize}
      \item<1-> All free-fermion states can be constructed by Dirac Hamiltonians,
      \[H(k)=iM+\left(\sum_{n=1}^d k_n^2\right)iM_0+\sum_{n=1}^d k_n\tilde{\Gamma}_n\]
      \item<1-> $\tilde \Gamma_m^2$ and $M$ form a representation of a Clifford Algebra
      \[-M^2=\tilde \Gamma_m^2=I; \{M,\tilde\Gamma_m\}=0;
      \{\tilde\Gamma_m,\tilde\Gamma_n\}=2\delta_{mn}.\]
      \item<2-> Classification problem: Can we find another mass term?
      \begin{itemize}
        \item Bulk: smooth variation of mass, $M(x)=M\cos x+M'\sin x$.
        \item Surface: existence of a mass term? can we gap it out?
      \end{itemize}
      \item<3-> $\mathrm{Cl}^{q+1,d}\rightarrow\mathrm{Cl}^{q+2,d}$:
      \[A(\mathrm{Cl}^{q+1,d}):=\frac{M(\mathrm{Cl}^{q+1,d})}{i^\ast M(\mathrm{Cl}^{q+2,d})}\simeq\pi_0(R^{q+1,d}).\]
    \end{itemize}
  \end{frame}

  \begin{frame}{Adding an onsite unitary symmetry}
    \begin{itemize}
      \item<1-> Why only TRS and PHS in 10-fold way?
      \item<2-> Free fermion + onsite unitary symmetry $G$: Hamiltonian is block-diagonalized into irreps of $G$:
      \[H=H_1\oplus H_2\oplus\cdots\begin{pmatrix}
        H_1 & 0 & 0\\
        0 & H_2 & 0\\
        0 & 0 & \cdots
      \end{pmatrix}\]
      \item<2-> Each block (each irrep) has a topological invariant.
      \item<3-> Real irreps: $\mathbb R$, $\mathbb C$ and $\mathbb H$-types.
      \item<3-> $\mathbb C$-type: $\mathrm{Cl}^{q+1,d}\rightarrow \mathbb{C}\mathrm{l}^{q+1+d}$;
      $\mathbb H$-type: $\mathrm{Cl}^{q+1,d}\rightarrow \mathrm{Cl}^{q+5,d}$.
    \end{itemize}
  \end{frame}

  \section{TCS with Point-Group Symmetries: Boundary States}

  \begin{frame}{Bulk Classification: Crystalline Equivalence Principle}
    \begin{itemize}
      \item<1-> Eyal Cornfeld and Adam Chapman, PRB \textbf{99}, 075105 (2019).
      \item<1-> Consider the Dirac Hamiltonian,
      \[H(k)=iM+\left(\sum_{n=1}^d k_n^2\right)iM_0+\sum_{n=1}^d k_n\tilde{\Gamma}_n\]
      \item<1-> Point-group action: $\hat g \tilde\Gamma_i\hat g^{-1}=R(g)_{ij}\tilde\Gamma_j.$
      \item<2-> Redefine $g=U_g[\tilde\Gamma_n]\hat g$: $g \tilde\Gamma_ig^{-1}=\tilde\Gamma_i.$
      \item<3-> Example: $U_{c_4}=e^{\tilde\Gamma_1\tilde\Gamma_2\frac\pi4}$.
      \item<4-> $\hat c_4^4=-1$ but $c_4^4=+1$. 

    \end{itemize}
  \end{frame}

  \begin{frame}{Anomaly Locations and Mass Fields}
    \begin{itemize}
      \item<1-> Example: $G=C_4$.
      \item<2-> Mass field: $M(x)$, $x\in X\subset S^2/G$; gapless region: $\bar X=(S^2/G) - X$.
      \item<2-> It is easier to study regions that can be gapped out.
      \item<3-> Deformation: G-equivariant map $X\rightarrow X'$.
      \begin{itemize}
        \item<3-> Minimal $X$: easy to study theoretically;
        \item<3-> Minimal $\bar X$: anomalous regions cannot be gapped out.
      \end{itemize}
      \item<4-> Results of $C_4$: three types of anomaly locations.
      \item<5-> Pure geometry: applies to all free-fermion symmetry classes / interacting fSPT.
    \end{itemize}
    \begin{center}
      \onslide<2->
      \begin{tikzpicture}[scale=1.2]
        \draw (0, 0) circle (1);
        %\draw (0, -1) to [out=30, in=-30] (0, 1) to [out=210, in=150] (0, -1);
        \filldraw [fill=black!30] (0, -1) to [out=30, in=-30] (0, 1) to [out=210, in=150] (0, -1);
        \fill [red] (0, 0.2) circle (0.3);
      \end{tikzpicture}
      \hspace{2em}
      \onslide<3->
      \begin{animateinline}{30}
        \multiframe{85}{Ra=0.05+0.01}{
          \begin{tikzpicture}[scale=1.2]
            \begin{scope}
            \path[draw, clip] (0, -1) to [out=30, in=-30] (0, 1) to [out=210, in=150] (0, -1);
            \fill[red] (-.6, -\Ra+0.1) to [out=-15, in=180+15] (.6, -\Ra+0.1) --
              (.6, \Ra+0.1) to [out=180+15, in=-15] (-.6, \Ra+0.1);
            \end{scope}
          \end{tikzpicture}
        }
      \end{animateinline}
      \hspace{4em}
      \onslide<4->
      \begin{tikzpicture}[scale=1.2]
        \fill [blue] (0, -1) circle (0.1);
        \fill [blue] (0, 1) circle (0.1);
        \path[draw, clip] (0, -1) to [out=30, in=-30] (0, 1) to [out=210, in=150] (0, -1);
        \draw[red, thick] (-.6, .05) to [out=-15, in=180+15] (.6, 0.05);
      \end{tikzpicture}
      \begin{tikzpicture}[scale=1.2]
        \draw (0, -1) to [out=30, in=-30] (0, 1) to [out=210, in=150] (0, -1);
        \draw [blue, thick] (0, -1) to [out=60, in=-60] (0, 1);
        \fill [red] (-0.2, 0) circle (0.1);
      \end{tikzpicture}
      \begin{tikzpicture}[scale=1.2]
        \fill [red] (0, 1) circle (0.1);
        %\fill [red] (0, -1) circle (0.1);
        \path[draw] (0, -1) to [out=30, in=-30] (0, 1) to [out=210, in=150] (0, -1);
      \end{tikzpicture}      
      \begin{tikzpicture}[scale=1.2]
        %\fill [red] (0, 1) circle (0.1);
        %\fill [red] (0, -1) circle (0.1);
        \path[draw, fill=blue] (0, -1) to [out=30, in=-30] (0, 1) to [out=210, in=150] (0, -1);
      \end{tikzpicture}
    \end{center}
  \end{frame}

  \begin{frame}{Classifying anomaly on the boundary}
    \begin{center}
      \begin{tikzpicture}[scale=1]
        %\fill [red] (0, 1) circle (0.1);
        %\fill [red] (0, -1) circle (0.1);
        \node at (-1.1, 0) {$X_1=$};
        \path[draw, fill=blue] (0, -1) to [out=30, in=-30] (0, 1) to [out=210, in=150] (0, -1);
      \end{tikzpicture}
      \hspace{2em}
      \begin{tikzpicture}[scale=1]
        \node at (-1.1, 0) {$X_2=$};
        \draw (0, -1) to [out=30, in=-30] (0, 1) to [out=210, in=150] (0, -1);
        \draw [blue, thick] (0, -1) to [out=60, in=-60] (0, 1);
        \fill [red] (-0.2, 0) circle (0.1);
      \end{tikzpicture}
      \hspace{2em}
      \begin{tikzpicture}[scale=1]
        \node at (-1.1, 0) {$X_3=$};
        \fill [blue] (0, -1) circle (0.1);
        \fill [blue] (0, 1) circle (0.1);
        \path[draw, clip] (0, -1) to [out=30, in=-30] (0, 1) to [out=210, in=150] (0, -1);
        \draw[red, thick] (-.6, .05) to [out=-15, in=180+15] (.6, 0.05);
      \end{tikzpicture}
      \hspace{2em}
      \begin{tikzpicture}[scale=1]
        \node at (-1.1, 0) {$X_4=$};
        \fill [red] (0, 1) circle (0.1);
        %\fill [red] (0, -1) circle (0.1);
        \path[draw] (0, -1) to [out=30, in=-30] (0, 1) to [out=210, in=150] (0, -1);
      \end{tikzpicture}      
    \end{center}

    \begin{itemize}
    \item $K_{X_i} := \{\text{States can be gapped out on } X_i\}$; $K=K_{X_1}\supset K_{X_2}\supset K_{X_3}\supset K_{X_4}$.
    \item $K_{X_1}/K_{X_2}$: strong TI; $K_{X_2}/K_{X_3}$: 1st-order; $K_{X_3}/K_{X_4}$: 2nd-order; $K_{X_4}$: atomic insulator.
    \item $K_{X_2}=A(\mathrm{Cl}^{q+1,3}\otimes\mathbb R[C_4])/A(\mathrm{Cl}^{q+1,3})$
    \item $K_{X_4}=A(\mathrm{Cl}^{q+1,3}\otimes\mathbb R[C_4])/A(\mathrm{Cl}^{q+1,3}\otimes\mathbb R[C_4])=0$. (No AI)
    \item $K_{X_3}$: whether $M$ and $c_4Mc_4^{-1}$ belong to the same conn. comp. in $R^{q+1, 3}$.
    \end{itemize}
  \end{frame}
  
  \begin{frame}{Example: class DIII}
    \begin{center}
      \begin{tikzpicture}[scale=1]
        %\fill [red] (0, 1) circle (0.1);
        %\fill [red] (0, -1) circle (0.1);
        \node at (-1.1, 0) {$X_1=$};
        \path[draw, fill=blue] (0, -1) to [out=30, in=-30] (0, 1) to [out=210, in=150] (0, -1);
      \end{tikzpicture}
      \hspace{2em}
      \begin{tikzpicture}[scale=1]
        \node at (-1.1, 0) {$X_2=$};
        \draw (0, -1) to [out=30, in=-30] (0, 1) to [out=210, in=150] (0, -1);
        \draw [blue, thick] (0, -1) to [out=60, in=-60] (0, 1);
        \fill [red] (-0.2, 0) circle (0.1);
      \end{tikzpicture}
      \hspace{2em}
      \begin{tikzpicture}[scale=1]
        \node at (-1.1, 0) {$X_3=$};
        \fill [blue] (0, -1) circle (0.1);
        \fill [blue] (0, 1) circle (0.1);
        \path[draw, clip] (0, -1) to [out=30, in=-30] (0, 1) to [out=210, in=150] (0, -1);
        \draw[red, thick] (-.6, .05) to [out=-15, in=180+15] (.6, 0.05);
      \end{tikzpicture}
      \hspace{2em}
      \begin{tikzpicture}[scale=1]
        \node at (-1.1, 0) {$X_4=$};
        \fill [red] (0, 1) circle (0.1);
        %\fill [red] (0, -1) circle (0.1);
        \path[draw] (0, -1) to [out=30, in=-30] (0, 1) to [out=210, in=150] (0, -1);
      \end{tikzpicture}      
    \end{center}

    \begin{itemize}
      \item Bulk classification: $\mathbb R[C_4]\simeq\mathbb R\oplus\mathbb R\oplus\mathbb C$;
      $A(\mathrm{Cl}^{2,3}\otimes\mathbb R[C_4])=\mathbb Z\oplus\mathbb Z\oplus\mathbb Z$.
      \item $K_{X_1}=\mathbb Z\oplus\mathbb Z\oplus\mathbb Z$.
      \item $K_{X_2}=A(\mathrm{Cl}^{q+1,3}\otimes\mathbb R[C_4])/A(\mathrm{Cl}^{q+1,3})=\mathbb Z\oplus\mathbb Z$.
      \item $K_{X_3}=2\mathbb Z\oplus\mathbb Z$.
      \item $K_{X_4}=0$.
      \item Strong TSC: $\mathbb Z$; 1st order: $\mathbb Z_2$; 2nd order: $2\mathbb Z\oplus\mathbb Z$; AI: 0.
    \end{itemize}
  \end{frame}

  \section{TCS with Space-Group Symmetries: Bulk Classification}

  \begin{frame}{Bulk Classification: Crystalline Equivalence Principle}
    \begin{itemize}
      \item<1-> Apply the Crystalline Equivalence Principle and see what we get ...
      \item<2-> $|SG|=\infty$: (unitary) irreps of space groups: labeled by momentum $\vec q$.\\
      \emph{$q^\ast$ + irreducible representation of the little group.}
      \item<3-> Example: $G=\mathbb Z$, unitary irreps are $g\mapsto e^{iq}$, labeled by $q\in [0,2\pi)\simeq\mathbb R/\mathbb Z$.
      \item<4-> Block-diagonalized Hamiltonian: $\hat H=\hat H_{q_1}\oplus\hat H_{q_2}\oplus\cdots$, \alert{finite} number of terms.
      \item<5-> $\hat H_q(k)$: do not confuse $q$ with $k$.
      \item<6-> Conjecture of equivalence mapping: $\hat H_{\vec q}(\vec k)=(k_i-q_i)\tilde\Gamma_i+iM$.
      \item<7-> Smooth variation of $k_i$ does not change the topological classification.
    \end{itemize}
    \onslide<7->
    \begin{center}
    \begin{animateinline}{30}
      \multiframe{100}{Ra=1.5+-0.01}{
    %\begin{animateinline}{5}
    %  \multiframe{10}{Ra=1.5+-0.1}{
      \begin{tikzpicture}[scale=2]
          \draw [->] (0, 0) -- (2, 0) node [right] {$k$};
          \draw [blue] (\Ra-0.3, -0.5) -- (\Ra+0.3, 0.5);
          \draw [blue] (\Ra+0.3, -0.5) -- (\Ra-0.3, 0.5);
        \end{tikzpicture}
      }
    \end{animateinline}      
    \end{center}
  \end{frame}

  \begin{frame}{Example: 1d, class BDI}
    \begin{itemize}
      \item<1-> $d=1$, class BDI ($\mathcal T^2=\mathcal C^2=+1$), $G=\mathbb Z$.
      \item<1-> Physically: spinless fermionic superconductor w/ TRS.
      \item<2-> $q\neq 0,\pi$: $V_q\simeq\mathbb C$: $t_x\mapsto\begin{pmatrix}\cos q & \sin q\\-\sin q & \cos q\end{pmatrix}$; $A(\mathbb C\mathrm l^{9})=\mathbb Z$ (SSH).
      \item<3-> $q=0,\pi$: $V_q\simeq\mathbb R$: $t_x\mapsto\pm1$; $A(\mathrm{Cl}^{8,1})=\mathbb Z$ (Kitaev chain).
      \item<4-> $q\rightarrow 0,\pi$: $\mathbb C\rightarrow2\mathbb R$: $\begin{pmatrix}\cos q & \sin q\\-\sin q & \cos q\end{pmatrix}\rightarrow\begin{pmatrix}\pm1&0\\0&\pm1\end{pmatrix}$.
      \item<5-> Classification: $\mathbb Z\oplus\mathbb Z\oplus\mathbb Z/\langle(2, -1, 0), (0, -1, 2)\rangle=\mathbb Z\oplus\mathbb Z_2$, generated by $(1, 0, 0)$ and $(1, 0, -1)$.
      \item<6-> $\mathbb Z$: strong TSC; $\mathbb Z_2$: weak TSC.
    \end{itemize}
    \onslide<2->
    \begin{center}
      \begin{tikzpicture}[scale=2]
        \draw [white] (0, 0) -- (0, 0.3);
        \draw [->] (0,0) -- (2, 0) node [right] {$q$};
        \fill (0, 0) circle (0.04);
        \node at (0, -0.05) [below] {0};
        \fill (1.6, 0) circle (0.04);
        \node at (1.6, -0.05) [below] {$\pi$};
        \node<4-> at (0, 0.2) {2};
        \node<4-> at (0.8, 0.2) {1};
        \node<4-> at (1.6, 0.2) {2};
        \draw<4-> [->] (0.7, 0.2) -- (0.1, 0.2);
        \draw<4-> [->] (0.9, 0.2) -- (1.5, 0.2);
      \end{tikzpicture}
    \end{center}
  \end{frame}

  \section{Summary}

  \begin{frame}{Summary and Outlooks}
    \begin{enumerate}
      \item TSC Bulk classification:
      \begin{itemize}
        \item Crystalline equivalence principle works for free fermions.
        \item Space-group TSC classification?
        \item Point-group TSC w/ arbitrary $\omega_2$? ($\omega_2$ from $1\rightarrow \mathbb Z_2^f\rightarrow G_f\rightarrow G_b\rightarrow0$.)
      \end{itemize}
      \item TSC surface states:
      \begin{itemize}
        \item Anomaly locations.
        \item Calculation for interacting fSPT and average SPT?
        \item Space-group SPT?
      \end{itemize}
    \end{enumerate}
  \end{frame}

\end{document}
