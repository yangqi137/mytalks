\documentclass[xcolor=table, 10pt, aspectratio=169]{beamer}

%\usepackage{arev}
\usepackage{amsmath,amssymb,amscd}
\usepackage{dsfont}
\usepackage{mathrsfs}
\usepackage{yfonts}
\usepackage{bm}
\usepackage{graphicx}
\usepackage{tabularx}
\usepackage{animate}
%\usepackage{mathtools}
%\usepackage{ifthen}

%\usepackage{xeCJK}
%\usepackage{fontspec}
%\newfontfamily\cjkfont{PingFang SC}
%\setCJKmainfont{PingFang SC}
\newcolumntype{x}{>{\centering\arraybackslash}X}
\renewcommand{\arraystretch}{1.5}
\newcommand{\uone}{\mathrm U(1)}
\DeclareMathOperator{\img}{img}

\usepackage{tikz}
	\usetikzlibrary{calc}
	\usetikzlibrary{arrows,shapes, positioning, matrix}
	\usetikzlibrary{decorations.markings}
	\tikzset{>=stealth}
	\tikzstyle arrowstyle=[scale=1]
	\tikzstyle directed=[postaction={decorate,decoration={markings,
 	   mark=at position .15 with {\arrow[arrowstyle]{stealth}}}}]
\tikzstyle string=[thick,postaction={decorate,decoration={markings,
    mark=at position .55 with {\arrow[arrowstyle]{stealth}}}}]
\tikzstyle dual_string=[dashed,postaction={decorate,decoration={markings,
    mark=at position .55 with {\arrow[arrowstyle]{stealth}}}}]

\tikzstyle dw=[thick,postaction={decorate,decoration={markings,
    mark=at position 1 with {\arrow[arrowstyle]{stealth}}}}]
\tikzstyle group=[mbg]
\newcommand*{\halfway}{0.5*\pgfdecoratedpathlength+.5*8pt}\tikzstyle arrowstyle=[scale=1]
\newcommand*{\halfwayb}{0.5*\pgfdecoratedpathlength}
\tikzstyle arrowstyle=[scale=1]
\tikzstyle fermion=[thick,postaction={decorate},decoration={markings,
    mark=at position \halfway with {\arrow[arrowstyle]{latex}}}]
\tikzstyle fermion2=[thick,postaction={decorate},decoration={markings,
        mark=at position \halfwayb with {\arrow[arrowstyle]{latex}}}]

\usepackage{pgffor}
\newcommand{\mb}[1]{\mathbf{#1}}
\renewcommand{\cal}[1]{\mathcal{#1}}

\newcommand{\ag}[2]{#1_\mb{#2}}
\newcommand{\cohosub}[1]{\scalebox{0.72}{\textswab{#1}}}
\newcommand{\cohosubsub}[1]{\scalebox{0.6}{\textswab{#1}}}
\newcommand{\coho}[1]{\textswab{#1}}

\DeclareMathOperator{\tr}{Tr}
\DeclareMathOperator{\im}{Im}
\DeclareMathOperator{\re}{Re}

\mode<presentation>
{
  %\usetheme{Warsaw}
  % or ...
  %\useoutertheme{rectangle}
  \setbeamertemplate{frametitle}[default][center]
  \defbeamertemplate{itemize item}{flat}{\begin{pgfpicture}{-1ex}{0ex}{1ex}{2ex}
      \pgfpathcircle{\pgfpoint{0pt}{.6ex}}{0.6ex}
      \pgfusepath{fill}
    \end{pgfpicture}%
  }
  \defbeamertemplate{itemize subitem}{flat}{\footnotesize\raise0.5pt\hbox{\textbullet}}
  \defbeamertemplate{itemize subsubitem}{flat}{\footnotesize\raise0.5pt\hbox{\textbullet}}

  %\useinnertheme{circles}
  \setbeamertemplate{items}[flat]
  \setbeamertemplate{sections/subsections in toc}[circle]
  \setbeamertemplate{blocks}[rounded]
  \setbeamertemplate{title page}[default][colsep=-4bp,rounded=true]
  \setbeamertemplate{part page}[default][colsep=-4bp,rounded=true]
  \setbeamercovered{transparent}
  %\usecolortheme{spruce}
  %\definecolor{THU}{RGB}{116,61,130}
  \definecolor{mbg}{RGB}{0,0,160}
  \setbeamercolor*{palette primary}{fg=white,bg=mbg}
  \setbeamercolor*{titlelike}{parent=palette primary}
  \setbeamercolor*{structure}{fg=mbg}
  \setbeamercolor{frametitle}{fg=white,bg=mbg}
  % or whatever (possibly just delete it)
  \setbeamercolor{block title}{bg=mbg,fg=white}
  \setbeamercolor{block body}{bg=mbg!15}


  \addtobeamertemplate{navigation symbols}{}{ \hspace{1em}%
    \usebeamerfont{footline}%
    \insertframenumber / \inserttotalframenumber }
}


%\usepackage[english]{babel}
% or whatever

%\usepackage[latin1]{inputenc}
% or whatever

%\usepackage{times}
%\usepackage[T1]{fontenc}
% Or whatever. Note that the encoding and the font should match. If T1
% does not look nice, try deleting the line with the fontenc.

\title[CP as Bdry] % (optional, use only with long paper titles)
{Report: Critical points as boundaries}

\author % (optional, use only with lots of authors)
{Wenjie Ji and Xiao-Gang Wen}
% - Give the names in the same order as the appear in the paper.
% - Use the \inst{?} command only if the authors have different
%   affiliation.

%\institute[Fudan] % (optional, but mostly needed)
%{Department of Physics, Fudan University}
% - Use the \inst command only if there are several affiliations.
% - Keep it simple, no one is interested in your street address.

%\date{2016 Annual Meeting of Fudan CFTPP} % (optional, should be abbreviation of conference name)
%{Fudan University, Oct 13 2015}
\date{Group meeting, Mar. 2020}
% - Either use conference name or its abbreviation.
% - Not really informative to the audience, more for people (including
%   yourself) who are reading the slides online

\subject{Theoretical Physics}
% This is only inserted into the PDF information catalog. Can be left
% out.



% If you have a file called "university-logo-filename.xxx", where xxx
% is a graphic format that can be processed by latex or pdflatex,
% resp., then you can add a logo as follows:

\pgfdeclareimage[height=1cm]{university-logo}{../resources/fudan}
\logo{\pgfuseimage{university-logo}}

\AtBeginSection[]
{
  \begin{frame}<beamer>{Outline}
			\tableofcontents[currentsection,currentsubsection]
  \end{frame}
}


% Delete this, if you do not want the table of contents to pop up at
% the beginning of each subsection:

\begin{document}

\begin{frame}
  \titlepage

  \begin{center}
    arXiv:1912.13492
  \end{center}
\end{frame}

\section{1d Transverse-Field Ising Model and $\mathbb Z_2\times\tilde{\mathbb Z}_2$ symmetry.}

\begin{frame}
  \frametitle{1d Transverse-Field Ising Model}
  \[H=-h\sum_i X_i-J\sum_i Z_iZ_{i+1}.\]
  \begin{itemize}
    \item $h/J=0$: Classical Ising Model, FM ground state.
    \item $J/h=0$: $X_i=+1$, PM ground state.
    \item Quantum Critical Point: @$h/J=1$, 2D-Ising universality class.
  \end{itemize}
  \begin{center}
    \begin{tikzpicture}
      \draw (-4, 0) -- (4, 0) [thick];
      \node at (4, 0) [right] {$h/J$};
      \fill circle [radius=.1];
      \node at (-2, 0) [above] {FM};
      \node at (2, 0) [above] {PM};
      \node at (0, -.1) [below] {QCP};
    \end{tikzpicture}
  \end{center}
\end{frame}

\begin{frame}
  \frametitle{Duality transformation}
  \begin{itemize}
    \item Spin configurations $\Rightarrow$ Domain-wall configurations.
    \item $\tilde X_{i+\frac12}=Z_iZ_{i+1}$: $\tilde X=1$ - no domain wall; $\tilde X=-1$ - domain wall.
    \item $\tilde Z$ flips $\tilde X$ and  creates a domain wall:
    $\tilde Z_{i-\frac12}\tilde Z_{i+\frac12} = X_i$.\\
    Flipping one spin creates two domain walls.
    \item Self-dual: $h/J\rightarrow J/h$; QCP @ the fixed point $h/J=1$.
  \end{itemize}
  \vspace{2em}
  \begin{columns}
    \column{.5\textwidth}
    \[H=-h\sum_i X_i-J\sum_i Z_iZ_{i+1}.\]
    \begin{center}
      \begin{tikzpicture}
        \draw (-2, 0) -- (2, 0) [thick];
        \node at (2, 0) [right] {$h/J$};
        \fill circle [radius=.1];
        \node at (-1, 0) [above] {FM};
        \node at (1, 0) [above] {PM};
        \node at (0, -.1) [below] {QCP};
      \end{tikzpicture}
    \end{center}
    \column{.5\textwidth}
    \[H=-h\sum_i \tilde Z_{i-\frac12}\tilde Z_{i+\frac12}-J\sum_i\tilde X_{i+\frac12}.\]
    \begin{center}
      \begin{tikzpicture}
        \draw (-2, 0) -- (2, 0) [thick];
        \node at (2, 0) [right] {$h/J$};
        \fill circle [radius=.1];
        \node at (-1, 0) [above] {$\widetilde{\text{PM}}$};
        \node at (1, 0) [above] {$\widetilde{\text{FM}}$};
        \node at (0, -.1) [below] {QCP};
      \end{tikzpicture}
    \end{center}
  \end{columns}
\end{frame}

\begin{frame}
  \frametitle{$\mathbb Z_2\times\tilde{\mathbb Z}_2$ symmetry}
  \begin{itemize}
    \item Original model: $\mathbb Z_2$ symmetry $Z_i\rightarrow -Z_i$; generated by $U=\prod_iX_i$.
    \item Dual model: $\tilde{\mathbb Z}_2$ symmetry $\tilde Z_{i+\frac12}\rightarrow -Z_{i+\frac12}$; generated by $\tilde U=\prod_i\tilde Z_{i+\frac12}$.
    \item The two phases break $\mathbb Z_2$ and $\tilde{\mathbb Z}_2$ symmetries, respectively.
  \end{itemize}
  \begin{center}
    \begin{tikzpicture}
      \draw (-4, 0) -- (4, 0) [thick];
      \node at (4, 0) [right] {$h/J$};
      \fill circle [radius=.1];
      \node at (-2, 0) [above] {$\text{FM} = \widetilde{\text{PM}}$};
      \node at (2, 0) [above] {$\text{PM} = \widetilde{\text{FM}}$};
      \node at (0, -.1) [below] {QCP};
    \end{tikzpicture}
  \end{center}
  \begin{itemize}
    \item QCP: gaplessness protected by $\mathbb Z_2\times\tilde{\mathbb Z}_2$ symmetry.
    \item Anomalous gapless state?
    \begin{itemize}
      \item Symmetry-protected gapless edges of SPT states (like TI).
      \item Chiral edge states of chiral intrinsic topological orders.
      \item LSM Theorem.
    \end{itemize}
    \item What is the bulk of this 1+1D QCP?
  \end{itemize}
\end{frame}

\section{1d TFIM as the boundary of 2d todic code}

\begin{frame}
  \frametitle{2d Toric-Code Topological Order}
  \begin{itemize}
    \item Four anyons: 1, $e$, $m$, and $f=e\times m$.
    \item $e$ and $m$ are bosons with a nontrivial braiding statistics; $f$ is a fermion.
    \item Two types of gapped edges: $e$ edge and $m$ edge.
    \item Neither $e$ nor $m$ condenses: a gapless edge.
  \end{itemize}
  \begin{columns}
    \column{.5\textwidth}
    \begin{center}
      \begin{tikzpicture}
        \draw (-2, 0) -- (2, 0) [thick];
        \fill circle [radius=.1];
        \node at (-1, 0) [above] {$\langle e\rangle\neq 0$};
        \node at (1, 0) [above] {$\langle m\rangle\neq0$};
        \node at (0, -.1) [below] {Gapless edge};
      \end{tikzpicture}
    \end{center}
    \column{.5\textwidth}
    \begin{center}
      \begin{tikzpicture}
        \draw (-2, 0) -- (2, 0) [thick];
        \node at (2, 0) [right] {$h/J$};
        \fill circle [radius=.1];
        \node at (-1, 0) [above] {$\text{FM}=\widetilde{\text{PM}}$};
        \node at (1, 0) [above] {$\text{PM}=\widetilde{\text{FM}}$};
        \node at (0, -.1) [below] {QCP};
      \end{tikzpicture}
    \end{center}
  \end{columns}
\end{frame}

\begin{frame}
  \frametitle{Wen-Plaquette Model}
  \begin{columns}
    \column{.3\textwidth}
    \begin{tikzpicture}
      \draw (0, 0)--(4, 0);
      \draw (0, 2)--(4, 2);
      \draw (0, 4)--(4, 4);
      \draw (0, 0)--(0, 4);
      \draw (2, 0)--(2, 4);
      \draw (4, 0)--(4, 4);
      \node at (1, 1) {A};
      \node at (3, 3) {A};
      \node at (1, 3) {B};
      \node at (3, 1) {B};
    \end{tikzpicture}
    \column{.7\textwidth}
    \begin{itemize}
      \item Wen-plaquette Model is equivalent to Kitaev's toric-code model:
      \[H=-\sum_A\bar X_i\bar X_j\bar X_k\bar X_l
      -\sum_B\bar Z_i\bar Z_j\bar Z_k\bar Z_l.\]
      \item Edge dof: $X=\bar X_i\bar X_{i+1}$ on B;
      $\tilde Z=\bar Z_i\bar Z_{i+1}$ on A.
      \item They are edge $e$ and $m$ excitations.
      \item $X$ and $\tilde Z$ does not commute with each other:
      $X_i=\tilde X_{i-\frac12}\tilde X_{i+\frac12}$;
      $\tilde Z_{i+\frac12}=Z_iZ_{i+1}$.
      \item Edge Hamiltonian: with $U=\infty$,
      \[H=-h\sum_iX_i-J\sum_i\tilde Z_{i+\frac12}
      -U\sum_i\left(\tilde X_{i-\frac12}X_i\tilde X_{i+\frac12}
      +Z_i\tilde Z_{i+\frac12}Z_{i+1}\right).\]
      \item Reduces to either
      $H=-h\sum_i X_i-J\sum_i Z_iZ_{i+1}$
      or
      $H=-h\sum_i \tilde Z_{i-\frac12}\tilde Z_{i+\frac12}-J\sum_i\tilde X_{i+\frac12}.$
    \end{itemize}
  \end{columns}
\end{frame}

\section{Consequences of the bulk-boundary correspondence}

\begin{frame}
  \frametitle{Bulk topological order and boundary CFT}
  \begin{center}
    \begin{tikzpicture}
      \draw (-2, 0) -- (2, 0) [thick];
      \fill circle [radius=.1];
      \node at (-1, 0) [above] {$\langle e\rangle\neq 0$};
      \node at (1, 0) [above] {$\langle m\rangle\neq0$};
      \node at (0, -.1) [below] {Gapless edge};
    \end{tikzpicture}
  \end{center}
  \begin{itemize}
    \item Gapped edges: described by Fusion category Rep$_{\mathbb Z_2}$ and Vec$_{\mathbb Z_2}$, respectively.
    \item Gapless edge: Ising CFT (w/ e-m duality symmetry).
  \end{itemize}
\end{frame}

\section{Generalization and open questions}

\begin{frame}
  \frametitle{Generalization to other phase transitions}
  \begin{itemize}
    \item Symmetry-breaking transitions: $\mathbb Z_2\rightarrow G$.
    \item Higher dimensions:
    \item QPT beyond symmetry breaking? Deconfined QCP? QCP w/ fermions?
  \end{itemize}
\end{frame}

\begin{frame}
  \frametitle{Correspondence b/w topological order and boundary CFT}
  \begin{itemize}
    \item Bulk-boundary correspondence between boundary CFT and bulk topological order:\\
    Liang Kong and Hao Zheng, arXiv:1905.04924 and 1912.01760
    \item The ``center'' of the boundary CFT is the bulk UMTC.
    \item This put constraints on possible boundary CFTs that can be realized for a given UMTC.
    \item Relation to bootstrap?
    \item More 1+1D examples?
    \item Bulk-boundary correspondence in higher dimensions?
  \end{itemize}
\end{frame}

\end{document}
