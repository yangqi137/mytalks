\documentclass[xcolor=table, 10pt, aspectratio=169, ignorenonframetext]{beamer}

%\usepackage{arev}
\usepackage{amsmath,amssymb,amscd}
%\usepackage{dsfont}
%\usepackage{mathrsfs}
%\usepackage{yfonts}
\usepackage{bm}
\usepackage{graphicx}
\usepackage{tabularx}
\usepackage{animate}
%\usepackage{mathtools}
%\usepackage{ifthen}

%\usepackage{xeCJK}
%\usepackage{fontspec}
%\newfontfamily\cjkfont{PingFang SC}
%\setCJKmainfont{PingFang SC}
\newcolumntype{x}{>{\centering\arraybackslash}X}
\renewcommand{\arraystretch}{1.5}

\usepackage{tikz}
	\usetikzlibrary{calc}
	\usetikzlibrary{arrows,shapes, positioning, matrix}
	\usetikzlibrary{decorations.markings}
	\tikzset{>=stealth}
	\tikzstyle arrowstyle=[scale=1]
	\tikzstyle directed=[postaction={decorate,decoration={markings,
 	   mark=at position .15 with {\arrow[arrowstyle]{stealth}}}}]
\tikzstyle string=[thick,postaction={decorate,decoration={markings,
    mark=at position .55 with {\arrow[arrowstyle]{stealth}}}}]
\tikzstyle dual_string=[dashed,postaction={decorate,decoration={markings,
    mark=at position .55 with {\arrow[arrowstyle]{stealth}}}}]

\tikzstyle dw=[thick,postaction={decorate,decoration={markings,
    mark=at position 1 with {\arrow[arrowstyle]{stealth}}}}]
\tikzstyle group=[mbg]
\usepackage{pgffor}
\usepackage{pgfplots}

\mode<presentation>
{
  %\usetheme{Warsaw}
  % or ...
  %\useoutertheme{rectangle}
  \setbeamertemplate{frametitle}[default][center]
  \defbeamertemplate{itemize item}{flat}{\begin{pgfpicture}{-1ex}{0ex}{1ex}{2ex}
      \pgfpathcircle{\pgfpoint{0pt}{.6ex}}{0.6ex}
      \pgfusepath{fill}
    \end{pgfpicture}%
  }
  \defbeamertemplate{itemize subitem}{flat}{\footnotesize\raise0.5pt\hbox{\textbullet}}
  \defbeamertemplate{itemize subsubitem}{flat}{\footnotesize\raise0.5pt\hbox{\textbullet}}

  %\useinnertheme{circles}
  \setbeamertemplate{items}[flat]
  \setbeamertemplate{sections/subsections in toc}[circle]
  \setbeamertemplate{blocks}[rounded]
  \setbeamertemplate{title page}[default][colsep=-4bp,rounded=true]
  \setbeamertemplate{part page}[default][colsep=-4bp,rounded=true]
  \setbeamercovered{transparent}
  %\usecolortheme{spruce}
  %\definecolor{THU}{RGB}{116,61,130}
  \definecolor{mbg}{RGB}{0,0,160}
  \setbeamercolor*{palette primary}{fg=white,bg=mbg}
  \setbeamercolor*{titlelike}{parent=palette primary}
  \setbeamercolor*{structure}{fg=mbg}
  \setbeamercolor{frametitle}{fg=white,bg=mbg}
  % or whatever (possibly just delete it)
  \setbeamercolor{block title}{bg=mbg,fg=white}
  \setbeamercolor{block body}{bg=mbg!15}


  \addtobeamertemplate{navigation symbols}{}{ \hspace{1em}%
    \usebeamerfont{footline}%
    \insertframenumber / \inserttotalframenumber }
}


%\usepackage[english]{babel}
% or whatever

%\usepackage[latin1]{inputenc}
% or whatever

%\usepackage{times}
%\usepackage[T1]{fontenc}
% Or whatever. Note that the encoding and the font should match. If T1
% does not look nice, try deleting the line with the fontenc.

\title[CNNMC] % (optional, use only with long paper titles)
{ML-assisted quantum many-body computation beyond Markov-Chain Monte Carlo}

\author[Y Qi] % (optional, use only with lots of authors)
{Yang~Qi}
% - Give the names in the same order as the appear in the paper.
% - Use the \inst{?} command only if the authors have different
%   affiliation.

\institute[Fudan] % (optional, but mostly needed)
{Department of Physics, Fudan University}
% - Use the \inst command only if there are several affiliations.
% - Keep it simple, no one is interested in your street address.

\date{International Conference on Quantum Artificial Intelligence\\Shanghai, July 9th, 2021.}
% - Either use conference name or its abbreviation.
% - Not really informative to the audience, more for people (including
%   yourself) who are reading the slides online

\subject{Theoretical Physics}
% This is only inserted into the PDF information catalog. Can be left
% out.



% If you have a file called "university-logo-filename.xxx", where xxx
% is a graphic format that can be processed by latex or pdflatex,
% resp., then you can add a logo as follows:

%\pgfdeclareimage[height=1cm]{university-logo}{fudan}
%\logo{\pgfuseimage{university-logo}}



% Delete this, if you do not want the table of contents to pop up at
% the beginning of each subsection:
\AtBeginSection[]
{
  \begin{frame}<beamer>{Outline}
			\tableofcontents[currentsection,currentsubsection]
  \end{frame}
}
%\AtBeginSubsection[]
%{
 % \begin{frame}<beamer>{Outline}
  %  \tableofcontents[currentsection,currentsubsection]
  %\end{frame}
%}


\begin{document}

\begin{frame}
  \titlepage
\end{frame}

\begin{frame}{References}
\begin{itemize}
%\item Works led by Liang Fu, Zi-Yang Meng, and Lei Wang.
\item Hong Kong Univ: Hong-Yu Lu, Zi-Yang Meng.
\item Institute of Physics, CAS: Chu-Hao Li.
\item Institute of Theoretical Physics, CAS: Wei Li.
\item Reference:\\
Hongyu Lu, Chuhao Li, Wei Li, YQ and Zi Yang Meng, arXiv:2106.00712
\begin{center}
	\includegraphics[height=2cm]{../people/weili}
	\includegraphics[height=2cm]{../people/ziyangmeng}
\end{center}
\end{itemize}
\end{frame}

\begin{frame}{Outline}
	%\begin{columns}
	%\column{.7\textwidth}
		\tableofcontents
  %\end{columns}
  % You might wish to add the option [pausesections]
\end{frame}

\section{Introduction to Markov Chain Monte Carlo: the Importance of Update}

\begin{frame}
  \frametitle{Monte Carlo simulation: an unbiased method}
  \begin{itemize}
    \item A widely used numerical method in statistical physics and quantum many-body physics.
    \item Unbiased: reliable statistical error bar.
    \item Fast: polynomial complexity.
    \item Universal: applies to any model without the sign program.
  \end{itemize}
\end{frame}

\begin{frame}
  \frametitle{Introduction to MCMC}
  \begin{itemize}
    \item Consider a statistical mechanics model:
    \[Z=\sum_{\mathcal C}e^{-\beta H[\mathcal C]} = \sum_{\mathcal C}W(\mathcal C).\]
    \item The Markov-chain Monte Carlo (MCMC) is a way to do importance sampling.
    \item A Markov chain is constructed,
    \[\cdots\rightarrow\mathcal C_{i-1}\rightarrow\mathcal C_i\rightarrow\mathcal C_{i+1}\rightarrow\cdots\]
    \item Markov chain: $p(\mathcal C_i\rightarrow\mathcal C_j)$ only depends on $\mathcal C_i$ (no memory).
    \item Goal: distribution of $\mathcal C$ converges to the Boltzmann distribution $W(\mathcal C)$.
    \item Any observable can be measured from a Markov chain,
    \[\langle O\rangle = \frac{\sum O(\mathcal C)W(\mathcal C)}{\sum W(\mathcal C)} \simeq
     \frac1{\mathcal N}\sum_iO(\mathcal C_i).\]
    \item Quantum Monte Carlo: a quantum statistical model is mapped to a classical statistical model.
  \end{itemize}
\end{frame}

\begin{frame}
  \frametitle{Detailed balance}
  \begin{itemize}
    \item Detailed balance:
    \[\mathcal C\leftrightarrow\mathcal D,\quad
		\frac{p(\mathcal C\rightarrow\mathcal D)}{p(\mathcal D\rightarrow\mathcal C)}=\frac{W(\mathcal D)}{W(\mathcal C)}.\]
    \item Detailed balance (and ergodicity) guarantees that if the MC converges, it converges to the desired distribution $W(\mathcal C)$.
    \item Metropolis-Hastings algorithm: propose -- accept/reject.
  \end{itemize}
\begin{columns}
\column{.5\textwidth}
\begin{tikzpicture}
\node at (0, 0) (C) [draw,circle] {$\mathcal C$};
\node<2-> at (3, 0) (D) [draw, rectangle] {$\mathcal D?$};
\node<2-> at (3, 1.5) (D1) [draw, rectangle] {$\mathcal D_1?$};
\node<2-> at (3, -1.5) (D2) [draw, rectangle] {$\mathcal D_2?$};
\node<2-> at (3, -2) {$\vdots$};
\draw<2-> [->] (C)--(D) node [midway, above] {{\small $q(\mathcal C\rightarrow\mathcal D)$}};
\draw<2-> [->] (C)--(D1);
\draw<2-> [->] (C)--(D2);
\node<3-> at (6, 1) (DD) [draw, circle] {$\mathcal D$};
\node<3-> at (6, -1) (DC) [draw, circle] {$\mathcal C$};
\draw<3-> [->] (D)--(DD) node [midway, above] {{\small $\alpha(\mathcal C\rightarrow\mathcal D)$}};
\draw<3-> [->] (D)--(DC);
\end{tikzpicture}
\column{.5\textwidth}
\begin{block}{Steps of Metropolis-Hastings algorithm}
\begin{enumerate}
\item<2-> Select a new state $\mathcal D_i$.
\item<3-> Accept or reject.
    \[p(\mathcal C\rightarrow\mathcal D) = q(\mathcal C\rightarrow\mathcal D)
    \alpha(\mathcal C\rightarrow\mathcal D).\]
    \[\alpha(\mathcal C\rightarrow\mathcal D) =
    \min\left\{1, \frac{W(\mathcal D)}{W(\mathcal C)}
    \frac{q(\mathcal D\rightarrow\mathcal C)}
    {q(\mathcal C\rightarrow\mathcal D)}\right\}.\]
\end{enumerate}
\end{block}
\end{columns}
\end{frame}

\begin{frame}
  \frametitle{Autocorrelation time}
  \begin{itemize}
    \item Autocorrelation time measures the efficiency of the update algorithm.
    \item ``time'' sequence:
    \[\cdots\rightarrow O(t-1)\rightarrow O(t)\rightarrow O(t+1)\rightarrow\cdots,
  \quad O(t) = O[\mathcal C(t)].\]
    \item Autocorrelation function
    \[\mathcal A_O(\Delta t)=\langle O(t)O(t+\Delta t)\rangle - \langle O(t)\rangle^2\propto e^{-\Delta t/\tau}.\]
  \end{itemize}
  \begin{columns}
    \column{.7\textwidth}
    \begin{block}{Complexity $\propto \tau$}
      \[\langle O\rangle = \frac1{\mathcal N}\sum_iO(\mathcal C_i).\]
      Statistical error $\delta O\sim\frac1{\sqrt{\mathcal N}}$ only if $O(i)$ are independent.

A statistically independent sample is generated in every $\tau$-steps.
\[\text{Complexity}=\text{Complexity of each step}\times\tau\times\mathcal N.\]
    \end{block}
    \column{.3\textwidth}
    \centering
    \includegraphics[height=3.5cm]{../slmctut/auto_decay}
  \end{columns}
\end{frame}

\begin{frame}
  \frametitle{Performence of the algorithm}
    \begin{columns}
			\column{.5\textwidth}
      \begin{center}
				Theory\\
				\vspace{.5cm}
        \includegraphics[height=4cm]{../slmctut/laptop_coffee}
      \end{center}
			\column{.5\textwidth}
      \begin{center}
				Reality\\
				\vspace{.5cm}
        \includegraphics[height=4cm]{../slmctut/tianhe}
      \end{center}
      %\item Why?
      %\[\text{Cost} \propto \text{Cost of each step} \times \text{autocorrelation time}\]
    \end{columns}
\vspace{.5cm}
Most of our simulation is done on the Tianhe-1 supercomputer in Tianjin, China.
\end{frame}

\begin{frame}
  \frametitle{A good update samples low-energy states}
  \begin{center}
    \begin{tikzpicture}
      \draw plot[domain=0:9.52,smooth,samples=200] function {sin(x*2)};
      \draw<1> (2.356, -1) -- (2.8, -0.63127) [->, thick];
      \draw<2> (2.356, -1) -- (3.5, 0.65699) [->, thick];
      \draw<3> (2.356, -1) -- (5.49779, -1) [->, thick];
    \end{tikzpicture}
  \end{center}
  \begin{itemize}
    \item A dilemma of local updates:
    \item<1-> Step is too small: high acceptance, small difference.
    \item<2-> Step is too big: low acceptance, big difference.
  \end{itemize}
\end{frame}

\begin{frame}
  \frametitle{Local Update: Metropolis algorithm}
  \begin{center}
    \begin{tikzpicture}
      \node at (0, 0) [circle, draw] {};
      \node at (1, 0) [circle, draw] {};
      \node at (0, 1) [circle, draw] {};
      \node at (-1, 0) [circle, fill, draw] {};
      \node at (0, -1) [circle, fill, draw] {};

      \node at (4, 0) [circle, fill, draw] {};
      \node at (5, 0) [circle, draw] {};
      \node at (4, 1) [circle, draw] {};
      \node at (3, 0) [circle, fill, draw] {};
      \node at (4, -1) [circle, fill, draw] {};

      \draw [<->, thick] (1.5, 0)--(2.5, 0);
    \end{tikzpicture}
  \end{center}
  \begin{itemize}
    \item Consider the Ising model.
    \item Local update: randomly select a site and flip the spin.
    \item $q(\mathcal C\rightarrow\mathcal D) = q(\mathcal D\rightarrow\mathcal C) = \frac1N$.
    \item $\alpha(\mathcal C\rightarrow\mathcal D)=\min\left\{1,\frac{W(\mathcal D)}{W(\mathcal C)}\right\}.$
    \item $N$ trials are counted as one MC step.
    \item Very general: applies to any model.
    \item N Metropolis, A W Rosenbluth, M N Rosenbluth, A H Teller, and E Teller, J Chem Phys \textbf{21}, 1087 (1953).
  \end{itemize}
\end{frame}

\begin{frame}
  \frametitle{Critical slowing down}
  \begin{itemize}
    \item The real dynamical relaxation time diverges at the critical point: a critical system is slow to equilibrate.
    \item The local update mimics the real relaxation process: also exhibits the critical slowing down phenomena.
    \item $\tau\propto L^z$, $z=2.125$ for the 2D Ising model.
  \end{itemize}
  \begin{center}
    \includegraphics[width=8cm]{slowdown}
  \end{center}
  There's a way around this: MCMC simulation does not have to mimic the real dynamics...
\end{frame}

\begin{frame}
  \frametitle{Challenge}
  \begin{itemize}
    \item Local update is too slow for many models: critical slowing down, glassy behavior, separation of energy scales, etc.
    \item Global update is only available for certain models. Like Wolff algorithm for two-body interactions.
    \item A good update algorithm for generic models?
  \end{itemize}
\end{frame}

\end{document}
